\documentclass[a4paper]{article}

\usepackage[english]{babel}
\usepackage[utf8]{inputenc}
\usepackage{amsmath}
\usepackage{amssymb}
\usepackage{graphicx}
\usepackage{ntheorem}
\usepackage[colorinlistoftodos]{todonotes}

\theoremstyle{break}
\newtheorem{ex}{{ \Large Esercizio} }
\theoremstyle{plain}
\newtheorem{sol}{Soluzione}[ex]

\title{Rappresentazioni \\ { \textit soluzioni agli esercizi per natale }}

\author{Second'anno di matematica, SNS}

\date{\today}

\begin{document}
\maketitle

\section*{Soluzioni agli esercizi}

\begin{ex} 


\begin{sol} 

\end{sol}

\begin{sol}

\end{sol}


\end{ex}

\begin{ex}


\begin{sol}

\end{sol}

\begin{sol}

\end{sol}


\end{ex}

\begin{ex}


\begin{sol}

\end{sol}

\begin{sol}

\end{sol}


\end{ex}

\begin{ex}


\begin{sol}

\end{sol}

\begin{sol}

\end{sol}


\end{ex}

\begin{ex}


\begin{sol}

\end{sol}

\begin{sol}

\end{sol}


\end{ex}

\begin{ex}


\begin{sol}

\end{sol}

\begin{sol}

\end{sol}


\end{ex}

\begin{ex}


\begin{sol}

\end{sol}

\begin{sol}

\end{sol}


\end{ex}

\begin{ex}


\begin{sol}

\end{sol}

\begin{sol}

\end{sol}


\end{ex}

\begin{ex}


\begin{sol}

\end{sol}

\begin{sol}

\end{sol}


\end{ex}

\newpage
\section*{Idee utili per gli esercizi}
\subsection*{Tabelle dei caratteri}

\end{document}
