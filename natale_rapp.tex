\documentclass[a4paper]{article}

\usepackage[english]{babel}
\usepackage[utf8]{inputenc}
\usepackage{amsmath}
\usepackage{amssymb}
\usepackage{graphicx}
\usepackage{ntheorem}
\usepackage[colorinlistoftodos]{todonotes}

\theoremstyle{break}
\newtheorem{ex}{{ \Large Esercizio} }
\theoremstyle{plain}
\newtheorem{sol}{Soluzione}[ex]

\title{Rappresentazioni \\ { \textit soluzioni agli esercizi per natale }}

\author{Second'anno di matematica, SNS}

\date{\today}

\begin{document}
\maketitle

\section*{Soluzioni agli esercizi}

\begin{ex} 
Siano $k$, e $n$ due interi positivi. Per ogni $\sigma \in S_k$ denotate con $\omega(\sigma)$ il numero di orbite di $\sigma$ su $\{1, \ldots, k\}$. Dimostrate la formula:
$$ \frac{1}{k!} \sum_{\sigma \in S_k} n^{\omega(\sigma)} = \binom{n+k-1}{k} $$

\begin{sol} 

\end{sol}

\begin{sol}

\end{sol}


\end{ex}

\begin{ex}
Calcolate la scomposizione in fattori irriducibili dei prodotti di tutte le possibili coppie di rappresentazioni irriducibili di $S_4$.

\begin{sol}

\end{sol}

\begin{sol}

\end{sol}


\end{ex}

\begin{ex}

\begin{itemize}
\item[(a)]  Sia $\rho: G \to \mathrm{GL}(V) $ una rappresentazione irriducibile di un gruppo $G$. Dimostrate che l’immagine del centro di G è contenuta nel sottogruppo dei multipli dell’identità.

\item[(b)] Dimostrate che ogni sottogruppo finito di $\mathbb{C}^*$ è ciclico

\item[(c)] Se un gruppo finito ha una rappresentazione fedele irriducibile, allora il suo

centro è ciclico. Nota: una rappresentazione $\rho$ di $G$ è fedele se $\rho: G \to \mathrm{GL}(V_{\rho}) $ è iniettivo.

\end{itemize}
\begin{sol}

\end{sol}

\begin{sol}

\end{sol}


\end{ex}

\begin{ex}
Se $\rho$ è una rappresentazione irriducibile di $S_5$ di grado 5 e $s \in S_5$ è un 5-ciclo, fate vedere che $\rho(s)$ ha come autovalori tutte e sole le radici quinte dell’unità.

\begin{sol}

\end{sol}

\begin{sol}

\end{sol}


\end{ex}

\begin{ex}
Trovare la tavola dei caratteri di $A_4$.

\begin{sol}

\end{sol}

\begin{sol}

\end{sol}


\end{ex}

\begin{ex}
Trovare la tavola dei caratteri di $D_4, D_5$.

\begin{sol}

\end{sol}

\begin{sol}

\end{sol}


\end{ex}

\begin{ex}
Sia $T$ un tetraedro di centro nell'origine, e sia $G \subseteq O_3$ il gruppo delle trasformazioni ortogonali che portano $T$ in se stesso. Numerando in qualche modo i vertici di $T$ da $1$ a $4$, otteniamo un’azione di $G$ su $\{1, 2, 3, 4\}$.
\begin{itemize}
\item[(a)] Fate vedere che questo dà un’identificazione di $G$ con $S_4$.

\item[(b)] Fate vedere che il sottogruppo di $G$ delle matrici con determinante positivo corrisponde ad $A_4$.

\item[(c)] Scomponete la rappresentazione per permutazioni corrispondente agli spigoli del tetraedro come rappresentazione di $A_4$.
\end{itemize}

\begin{sol}

\end{sol}

\begin{sol}

\end{sol}


\end{ex}

\begin{ex}


\begin{sol}

\end{sol}

\begin{sol}

\end{sol}


\end{ex}

\begin{ex}


\begin{sol}

\end{sol}

\begin{sol}

\end{sol}


\end{ex}

\newpage
\section*{Idee utili per gli esercizi}
\subsection*{Tabelle dei caratteri}

\end{document}
